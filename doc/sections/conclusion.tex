\section{Conclusion}\label{sec:conclusion}

This research presents a groundbreaking approach to theological questions through quantum computation. Our quantum simulation of contingent reality has yielded significant insights into the necessity of a self-existent being.

\subsection{Key Findings}
Our results demonstrate:
\begin{itemize}
    \item High system entropy (4.971) showing inherent instability
    \item Low quantum coherence (-0.912) indicating dependency
    \item Significant causal relationships (0.488)
    \item Overall necessity score of 81.8\%
\end{itemize}

\subsection{Implications}
These findings support the Leibnizian Contingency Argument by:
\begin{enumerate}
    \item Demonstrating computational instability of purely contingent systems
    \item Showing mathematical necessity of a non-contingent foundation
    \item Providing empirical support for theological arguments
    \item Bridging quantum mechanics and metaphysics
\end{enumerate}

\subsection{Contributions}
This work advances multiple fields:
\begin{itemize}
    \item \textbf{Quantum Computing:} Novel theological applications
    \item \textbf{Theoretical Physics:} Quantum-modal logic connections
    \item \textbf{Computational Theology:} First quantum-based proof
\end{itemize}

\subsection{Future Work}
Promising directions include:
\begin{itemize}
    \item Increasing system complexity (more qubits)
    \item Implementing on real quantum hardware
    \item Exploring additional theological arguments
    \item Developing practical applications
\end{itemize}

This research establishes a new paradigm for investigating theological questions through computational means, opening avenues for future exploration at the intersection of quantum mechanics, mathematics, and theology.